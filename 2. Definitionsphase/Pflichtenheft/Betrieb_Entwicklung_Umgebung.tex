\section{Ressourcen}

\subsection{Entwickler}
Es gibt drei technische Themenbereiche, die für die Webseite abgedeckt werden müssen:
\begin{itemize}
	\item \textbf{HTML / CSS} \\
	Struktur und Gestaltung der Webseite
	\item \textbf{JSP / Java Beans} \\
	Serverseitige Logik der Webseite
	\item \textbf{MySQL} \\
	Speichern und Laden der Benutzerdaten (wie Karteikarten)
\end{itemize}
Dieses Projekt wird durch drei Entwickler unterstützt, die jeweils auf eins dieser Themen ihren Schwerpunkt haben.

\subsection{Server}
unbeaufsichtigter Betrieb
Die Daten zum benutzten Server:
\begin{itemize}
	\item \textbf{Ubuntu Server 13.10} \\
	Ein Debian-basiertes Linux-Betriebssystem, der für den Server-Betrieb gedacht ist. \\
	\textit{Lizenz:} GNU GPL (und andere)
	\item \textbf{Apache Tomcat 7} \\
	Ein Web Server und Servlet Container. Es unterstützt die Java Servlet  und JSP-Specification. \\
	\textit{Lizenz:} Apache License 2.0
\end{itemize}

\subsection{Verwendete Libraries}
\begin{itemize}
	\item \textbf{JSP} \\
	Web-Programmiersprache zur dynamischen Erzeugung von HTML. Sie erlaubt die Einbettung von Java und JSP-Aktionen im HTML. \\
	\textit{Lizenz:} GNU GPL v2 und CDDL v1.1
	\item \textbf{jQuery} \\
	Javascript-Bibliothek zur DOM-Navigation und -Modifikation.\\
	\textit{Lizenz:} MIT
	\item \textbf{TinyMCE} \\
	Ein auf JavaScript basierter freier WYSIWYG-Editor, der mit dem EqnEditor\_tinymce Plugin auch mathematische Formeln erstellen kann. \\
	\textit{Lizenz:} LGPL
	\item \textbf{Font Awesome} \\
	Skalierbare Vektorbilder, die über CSS angepasst werden können. \\
	\textit{Lizenz:} MIT
\end{itemize}

