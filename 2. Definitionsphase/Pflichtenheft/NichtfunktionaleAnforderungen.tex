\section{Nichtfunktionale Anforderungen}
Diese Anforderungen befassen sich nicht nicht mit der Funktionalität des Projektes, sondern mit anderen Themen wie z.B. Qualität, Schnelligkeit und Benutzerfreundlichkeit. Auch rechtliche Anforderungen können hier beschrieben werden.

\subsection{Qualität}
\begin{description}
     \item[/NF10/] Die Webseite soll die Musskriterien erfüllen.
     \item[/NF20/] Die Webseite soll möglichst unbeaufsichtigt betrieben werden können.
     \item[/NF30/] Das Lernen der Karteikarten soll möglichst einfach sein.
     \item[/NF40/] Fehler ausgelöst durch den Benutzer soll abgefangen werden und wenn nötig durch eine Fehlermeldung angezeigt werden.
     \item[/NF50/] Die Seite soll ein Stresstest mit 300 registrierten Benutzern und 200 Aufrufe in der Minute aushalten.
\end{description}

\subsection{Getestete Browser}
\begin{itemize}
	\item Google Chrome (ab Version ...)
	\item Mozilla Firefox (ab Version ...)
	\item Internet Explorer (ab Version ...)
	\item Apple Safari (ab Version ...)
	\item Google Chrome for Android (ab Version ...)
	\item Apple Safari for iPad (ab Version ...)
\end{itemize}

\subsection{Rechtliche Informationen}
Das Produkt ist quelloffen und wird nichtkommerziell unter der freien "GNU General Public Licence"-Lizenz zur Verfügung gestellt. Dies ist mit allen verwendeten Bibliotheken und Tools vereinbar.
