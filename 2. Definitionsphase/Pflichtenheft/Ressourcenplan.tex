\section{Ressourcenplan}
\subsection{Mitarbeiter}
\begin{center}
\begin{tabulary}{\textwidth}{|L|L|L|L|L|}
\hline 
\textbf{Nr} & \textbf{Name} & \textbf{Zuständigkeit} & \textbf{Arbeitstunden pro Tag\newline (verfügbar / effektiv)}\\  
\hline
1 & David Ehlen & Softwareentwickler \& UX-Designer & 8 / 2\\ 
\hline
2 & Julien Hadley Jack & Softwareentwickler \& Server-Administrator & 8 / 2\\ 
\hline
3 & Sebastian Dernbach & Softwareentwickler \& Projektleiter & 8 / 2\\ 
\hline
%&&&&& \\ 
%\hline 
\end{tabulary} 
\end{center}

\noindent Da alle drei Mitarbeiter während der Projektzeit die Duale Hochschule besuchen, kann in diesem Zeitraum nur mit einer Arbeitszeit von 2 Stunde pro Tag gerechnet werden.

\subsection{Hardware}
Alle notwendigen Computer, die zur Planung und Entwicklung nötig sind, werden schon vor Projektstart von den Mitarbeitern zur Verfügung gestellt.

%\subsubsection{Entwicklung}
%Es gibt vier technische Themenbereiche, die für die Webseite abgedeckt werden müssen:
%\begin{itemize}
%	\item \textbf{HTML / CSS} \\
%	Struktur und Gestaltung der Webseite
%	\item \textbf{Javascript / jQuery}
%	Klientseitige Interaktion mit der Webseite
%	\item \textbf{JSP / Java Beans} \\
%	Serverseitige Logik der Webseite
%	\item \textbf{MySQL} \\
%	Speichern und Laden der Benutzerdaten (wie Karteikarten)
%\end{itemize}

\subsubsection{Server}
Der Server läuft im unbeaufsichtigten Betrieb. Die Daten zum benutzten Server:
\begin{itemize}
	\item \textbf{Ubuntu Server 13.10} \\
	Ein Debian-basiertes Linux-Betriebssystem, der für den Server-Betrieb gedacht ist. \\
	\textit{Lizenz:} GNU GPL (und andere)
	\item \textbf{Apache Tomcat 7} \\
	Ein Web Server und Servlet Container. Er unterstützt die ``Java Servlet``-  und JSP-Spezifikation. \\
	\textit{Lizenz:} Apache License 2.0
\end{itemize}

\subsection{Software}
\subsubsection{Verwendete Libraries}
\begin{itemize}
	\item \textbf{JSP} \\
	Web-Programmiersprache zur dynamischen Erzeugung von HTML. Sie erlaubt die Einbettung von Java und JSP-Aktionen im HTML. \\
	\textit{Lizenz:} GNU GPL v2 und CDDL v1.1
	\item \textbf{jQuery} \\
	Javascript-Bibliothek zur DOM-Navigation und -Modifikation.\\
	\textit{Lizenz:} MIT
	\item \textbf{TinyMCE} \\
	Ein auf JavaScript basierter freier WYSIWYG-Editor, der mit dem ``EqnEditor\-TinyMCE``-Plugin auch mathematische Formeln erstellen kann. \\
	\textit{Lizenz:} LGPL
	\item \textbf{Font Awesome} \\
	Skalierbare Vektorbilder, die über CSS angepasst werden können. \\
	\textit{Lizenz:} MIT
\end{itemize}