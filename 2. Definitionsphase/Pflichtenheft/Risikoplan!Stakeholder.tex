%\section{Risikoplan}
%Probleme, die bei diesem Projekt entstehen können:
%
%\begin{description}
%	\item[Anforderungen] 
%	\item[Qualität] 
%	\item[Ressourcen]
%	\item[Krankheit]
%	\item[Projektende] Bei einer fehlerhaften Planung kann 
%	
%\end{description}

\newgeometry{margin=2cm} % modify this if you need even more space
\begin{landscape}
\begin{center}
\section{Risikoplan}
Probleme, die bei diesem Projekt entstehen können:
\medskip

\begin{tabulary}{1.35\textwidth}{|L|L|L|L|L|L|L|}
\hline 
\textbf{Risiko} & \textbf{Eintretender Schaden} & \textbf{Eintrittswahrscheinlichkeit} & \textbf{Schadenshöhe} & \textbf{Gegenmaßnahme} \\ 
\hline 
Nicht abgefangene Fehler & Verwirrung beim Benutzer, Inkorrekter Zustand der Benutzerdaten & mäßig & hoch & Stresstests, Testen durch viele Personen\\ 
\hline
Krankheit & Verlust von 30\% der Arbeitskraft & mäßig & sehr hoch & Puffer nach hinten einplanen \\ 
\hline
%&&&& \\ 
%\hline  
\end{tabulary}

\bigskip
\section{Stakeholder-Analyse}
\begin{tabulary}{1.35\textwidth}{|L|L|L|C|C|L|}
\hline 
\textbf{Gruppe} & \textbf{Interesse} & \textbf{Erwartung / Befürchtung} & \textbf{Stimmung / Klima} & \textbf{Macht / Einfluss} & \textbf{Maßnahmen} \\ 
\hline 
Wettbewerber & Erfolg des Produktes & erhöhter Druck auf eigenes Produkt & -1 & 1 & Kunden gewinnen durch Werbekampagnen \\ 
\hline 
Kunden & Produkt konsumieren & Erfüllung der eigenen Bedürfnisse & 1 & 5 & Anforderungen nach Kundenwünschen, Zwischenstände an Kunden testen \\ 
\hline 
Projektleiter & Erfolgreiches Projekt & Erfahrung durch Leitung des Projektes & 1 & 5 & mehr Verantwortung geben \\ 
\hline 
Mitarbeiter & Erfolgreiches Abschließen eigener Aufgaben & Angst: schlechte Projektplanung & 0 & 4 & In Planung, wenn nötig, einbeziehen \\ 
\hline
%&&&&& \\ 
%\hline 
\end{tabulary} 
 
\end{center}


\end{landscape}
\restoregeometry