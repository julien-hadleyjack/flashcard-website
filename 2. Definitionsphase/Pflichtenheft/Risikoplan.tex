%\section{Risikoplan}
%Probleme, die bei diesem Projekt entstehen können:
%
%\begin{description}
%	\item[Anforderungen] 
%	\item[Qualität] 
%	\item[Ressourcen]
%	\item[Krankheit]
%	\item[Projektende] Bei einer fehlerhaften Planung kann 
%	
%\end{description}

\newgeometry{margin=2cm} % modify this if you need even more space
\begin{landscape}
\begin{center}
\section{Risikoplan}
Probleme, die bei diesem Projekt entstehen können:

\begin{tabulary}{1.35\textwidth}{|L|L|L|L|L|L|L|}
\hline 
\textbf{Risiko} & \textbf{Eintretender Schaden} & \textbf{Eintrittswahrscheinlichkeit} & \textbf{Schadenshöhe} & \textbf{Gegenmaßnahme} \\ 
\hline 
Nicht abgefangene Fehler & Verwirrung beim Benutzer, Inkorrekter Zustand der Benutzerdaten & mäßig & hoch & Stresstests, Testen durch viele Personen\\ 
\hline
Krankheit & Verlust von 30\% der Arbeitskraft & mäßig & sehr hoch & Puffer nach hinten einplanen \\ 
\hline
&&&& \\ 
\hline  
\end{tabulary}

\section{Stakeholder-Analyse}
\begin{tabulary}{1.35\textwidth}{|L|L|L|L|L|L|}
\hline 
\textbf{Gruppe} & \textbf{Interesse} & \textbf{Erwartung / Befürchtung} & \textbf{Stimmung / Klimma} & \textbf{Macht / Einfluss} & \textbf{Maßnahmen} \\ 
\hline 
Wettbewerber &&&&&\\ 
\hline 
Kunden &&&&&\\ 
\hline 
Projektleiter &&&&& \\ 
\hline 
Mitarbeiter &&&&& \\ 
\hline 
\end{tabulary} 
 
\end{center}


\end{landscape}
\restoregeometry