\section{Zielbestimmung}

\subsection{Musskriterien}
Diese Kriterien müssen bis zum Projektende implementiert werden, damit das Projekt erfolgreich abgeschlossen ist.

\begin{description}
	\item[/MK10/] Die zu lernenden Karten werden durch einen Algorithmus ausgewählt (z.B. Leitner System)
	\item[/MK20/] Unicode-Zeichen werden als Eingabe in den Karteikarten unterstützt.
	\item[/MK30/] Bilder können eingefügt werden.
	\item[/MK40/] Ein Karteikartenset kann vom Benutzer exportiert werden.
	\item[/MK50/] Ein (anderer) Benutzer kann das vom \textbf{/MK40/} exportierte Karteikartenset für seinen Account importieren.
	\item[/MK60/] Die Seite stellt ein Katalog von vorgefertigten Karteikartensets zur Verfügung.
	\item[/MK70/] Der Benutzer hat Option ein Karteikartenset mit einer anderen Person zu teilen.
	\item[/MK80/] Beim Lernen mit den Karteikarten kann auch nur mit der Tastatur navigiert werden.
\end{description}

\subsection{Wunschkriterien}
Diese Kriterien werden für das Projekt gewünscht, sind aber nicht zwingend erforderlich für ein erfolgreiches Projektende. Je nach Zeit und technisches Schwierigkeitsgrad wird versucht möglichst viele/wichtige Kriterien zu implementieren.

\begin{description}
	\item[/WK10/] Die Webseite wird zusätzlich auch für Mobilgeräte optimiert.
	\item[/WK20/] Für jedes Karteikartenset werden Statistiken angezeigt (Erfolgsquote,...).
	\item[/MK30/] Mathematische Ausdrücke werden dargestellt in natürlicher Form (z.b. \( e^{\frac{3}{4}\pi i}\))
	\item[/MK40/] Die Antwortkarte kann in Form von Multiple Choice sein.
\end{description}

\subsection{Abgrenzkriterien}
Diese Kriterien beschreiben die Grenzen des Projektes und sollen auf jeden Fall nicht implementiert werden.
\begin{description}
	\item[/AK10/] Videos müssen nicht in die Karten eingebettet sein. (Sie können trotzdem einfach als Hyperlink eingefügt werden.)
	\item[/AK20/] Die Webseite muss nicht offline funktionieren.
	\item[/AK30/] Bei Verbindungsverlust muss der aktuelle Stand nicht zwischengespeichert werden.
\end{description}