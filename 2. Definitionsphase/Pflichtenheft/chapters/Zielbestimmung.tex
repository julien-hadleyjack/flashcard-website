\section{Zielbestimmung}

\subsection{Musskriterien}
\begin{description}
	\item[/MK10/] Algorithmen wählen die zu lernenden Karteien aus (z.B. Leitner System)
	\item[/MK20/] Unicode wird als Eingabe in den Karteikarten unterstützt.
	\item[/MK30/] Bilder können eingefügt werden.
	\item[/MK40/] Mathematische Ausdrücke werden dargestellt in natürlicher Form (z.b. \( e^{\frac{3}{4}\pi i}\))
	\item[/MK50/] Ein Karteikartenset kann vom Benutzer exportiert werden.
	\item[/MK60/] Ein Benutzer kann das vom \textbf{/MK40/} exportierte Karteikartenset für seinen Account importieren.
	\item[/MK70/] Die Antwortkarte kann in Form von Multiple Choice sein.
	\item[/MK80/] Die Seite stellt ein Katalog von vorgefertigten Karteikartensets zur Verfügung.
	\item[/MK90/] Der Benutzer hat Option ein Karteikartenset mit einer anderen Person zu teilen.
	\item[/MK100/] Beim Lernen mit den Karteikarten kann auch nur mit der Tastatur navigiert werden.
\end{description}

\subsection{Wunschkriterien}
\begin{description}
	\item[/WK10/] Die Webseite wird zusätzlich auch für Mobilgeräte optimiert.
\end{description}

\subsection{Abgrenzkriterien}